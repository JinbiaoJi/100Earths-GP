\documentclass{aastex6}
\usepackage{graphicx}
\usepackage{mathrsfs,amssymb}
\usepackage{amsmath}
\shorttitle{No evidence for planets orbiting Kapteyn's star}
\shortauthors{Us}

\begin{document}
\pagestyle{plain}
\pagenumbering{arabic}

\title{A Gaussian Process Regression Reveals No Evidence for Planets
Orbiting Kapteyn's Star}

\author{Author 1\altaffilmark{1}}
\author{Author 2\altaffilmark{1}}
\author{etc\altaffilmark{1}}

\altaffiltext{1}{University of Delaware \\ Department of Physics
and Astronomy \\ 217 Sharp Lab \\ Newark, DE 19716 USA}

\email{someone@udel.edu}

\begin{abstract}

This is the abstract

\end{abstract}

\keywords{planets and satellites: detection --- stars: rotation
--- stars: activity --- stars: individual (GJ 191) --- methods:
statistical}

\maketitle

\section{Introduction}

Why did we do the project? What's the interesting science
question?

A bit of info about Kapteyn's star

\section{Gaussian process model}

High-level description of how it works

Equations describing kernels

Explanation of likelihood function

Convergence tests

\section{Results: Rotation and search for residual periodicity}

Selecting appropriate model complexity

Posteriors

Realizations of the GP

Reduced $\chi^2$ comparison between 2-planet model and GP model

Periodograms of residuals

\section{Discussion and conclusions}

What did we learn?

What are the implications for future exoplanet research?

Acknowledgments: This research was supported by ... We thank
Dan Foreman-Mackey, Jessi Cisewski and others for their help ...

\begin{thebibliography}{99} 

% \bibitem[Aikawa et al.(1997)]{aikawa97} Aikawa, Y., Umebayashi,
% T., Nakano, T., \& Miyama, S.~M.\ 1997, \apj, 486, L51

% \bibitem[Ansdell et al.(2017)]{ansdell17} Ansdell, M., Williams,
% J.~P., Manara, C.~F., et al.\ 2017, \aj, 153, 240

\end{thebibliography}

%%%%%%%%%%%% FIGURES %%%%%%%%%%%%%%%%%

\clearpage

% Syntax for including a figure
% \begin{figure}[ht]
% \centering
% \includegraphics[width=0.90\textwidth]{fCinCO.pdf} 
% \caption{Fraction of carbon atoms incorporated in CO as a
% function of time.}
% \label{fig:fcovstime}
% \end{figure}

% Note: you can also put the figures directly in the main text

\end{document}                     

