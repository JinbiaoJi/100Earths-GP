\documentclass{aastex6}
\usepackage{graphicx}
\usepackage{mathrsfs,amssymb}
\usepackage{amsmath}
\shorttitle{No evidence for planets orbiting Kapteyn's star}
\shortauthors{Us}

\begin{document}
\pagestyle{plain}
\pagenumbering{arabic}

\title{A Gaussian Process Regression Reveals No Evidence for Planets
Orbiting Kapteyn's Star}

\author{Anna Bortle\altaffilmark{1}}
\author{Hallie Fausey\altaffilmark{1}}
\author{Jinbiao Ji\altaffilmark{1}}

\altaffiltext{1}{University of Delaware \\ Department of Physics
and Astronomy \\ 217 Sharp Lab \\ Newark, DE 19716 USA}

\email{abortle@udel.edu}

\begin{abstract}

  Radial-velocity (RV) planet searches are often polluted by signals caused by gas motion at the star's surface. Stellar activity can mimic or mask changes in the RVs caused by orbiting planets, resulting in false positives or missed detections.
  Here we use Gaussian process (GP) regression to disentangle the contradictory report of planets vs. rotation artifacts in Kapteyn's star (Anglada-Escud\'e et al. 2014, Robertson et al. 2015, Anglada-Escud\'e et al. 2016).
  To model rotation, we use joint quasi-periodic kernels for the RV and H$\alpha$ signals, requiring that their periods and decorrelation timescales be the same.
  We then construct residual RV data sets by subtracting off realizations of the GP drawn from Markov-chain Monte Carlo sampling.
  Having removed the velocity signature of rotation, we calculate the periodogram of each realization of the residuals to determine if any significant signals remain that may suggest an exoplanet.
  We conclude that the periodic signals of both previously reported ``planets'' are products of the star's rotation and activity.

\end{abstract}

\keywords{planets and satellites: detection --- stars: rotation
--- stars: activity --- stars: individual (GJ 191) --- methods:
statistical}

\maketitle

\section{Introduction}\label{sec:intro}

\textbf{Why did we do the project? What's the interesting science
question?}

\textbf{A bit of info about Kapteyn's star}\\

In 2014, two exoplanets were reported around Kapteyn's star (HD 33793), a red M1 sub-dwarf located 3.91 parsecs from the Sun.
The two planets, Kapteyn b and c, have reported periods of 48.6 and 121.5 days, respectively. The existences of the planets were contested in 2015 by Robertson et al..
Robertson claimed that Kapteyn's stellar rotation period is 143 days and that the 48.6 day period associated with Kapteyn b is an integer fraction (1/3) of this rotation and therefore is a product of stellar activity.
Additionally, the period of 121.5 days is close to that of the star's rotation which necessitates further observation of the star to determine the validity of Kapteyn c. \par
With debate on the rotation period of Kapteyn's star and to what extent its rotation and stellar activity affect the radial velocity signals, we look to use Gaussian Processes to jointly model the star's rotation and activity. After subtracting the model from the observed data, the residuals will contain exclusively any signals from possible exoplanets.

\section{Gaussian process model}\label{sec:GP}

\textbf{High-level description of how it works}

\textbf{Equations describing kernels}

\textbf{Explanation of likelihood function}

\textbf{Convergence tests}

\begin{figure} \label{fig:data}
    \begin{center}
    \includegraphics[width = 0.8\textwidth, height = 10cm]{figures/data.pdf}
    \caption{Radial velocity and H$\alpha$ measurements of Kapteyn's Star from HARPS and HIRES spectrographs. }
     \end{center}
\end{figure}

The GP works by using covariance matrices to describe how closely an RV or H$\alpha$ data point is related to other data points at various times in the series. For our model, we included a periodic kernel for the stars rotation and a squared exponential kernel to account for the dynamic configuration of stellar activity. Such activity would cause a weaker correlation between data points that are farther from each other in the time series. Multiplying the two kernels together and adding a white noise term, accounting for the observed data uncertainties, results in the final quasiperiodic kernel used to model the system. The white noise term adds the uncertainties to the diagonals of the covariance matrix.
\begin{figure} \label{fig:kernel_visual}
    \begin{center}
    \includegraphics[width = 0.9\textwidth]{figures/kernels.pdf}
    \caption{Images of each component of the quasiperiodic kernel.}
     \end{center}
\end{figure}\\

The equation describing the quasiperiodic kernel is as follows:
\begin{equation} \label{eq:quasi_kernel}
  k_{i,j} = A\ \mathrm{exp}\left[-\frac{(t_i - t_j)^2}{\lambda^2} - \frac{1}{2\omega^2}\mathrm{sin}^2\left(\frac{\pi(t_i -t_j)}{P}\right)\right] + \sigma_i^2\delta_{ij},%% <-- potentially cut the noise term of the equation
  \end{equation}
where $A$ is the amplitude, $\lambda$ is the decorrelation time-scale, $\omega$ is the smoothness of the periodic component, $P$ is the rotation period, and $\sigma$ is the white noise.\par
 %The uncertainties of each data point were accounted for in George's compute function, which treats them as white noise, and adds them the the diagonals of the covariance matrices in quadrature.\par
 When creating the possible models, we considered the possibility that a long-term activity trend existed in the time interval covered by the given data set.
 Thus we had two models: one with just the quasiperiodic kernel given by Eq. \ref{eq:quasi_kernel}, and one with an additional exponential squared kernel and an associated amplitude parameter applied to $H\alpha$ to model long term signal drift given by
 \begin{equation} \label{eq:ha_lt_kernel}
 k_{i,j} = A\ \mathrm{exp}\left[-\frac{(t_i - t_j)^2}{\lambda^2} - \frac{1}{2\omega^2}\mathrm{sin}^2\left(\frac{\pi(t_i -t_j)}{P}\right)\right] + B\ \mathrm{exp}\left[-\frac{(t_i - t_j)^2}{\gamma^2}\right] + \sigma_i^2\delta_{ij},
 \end{equation}
 
 where B is the amplitude and $\gamma$ is the decorrelation time-scale for the long-term activity trend in H$\alpha$.


\section{Results: Rotation and search for residual periodicity}\label{sec:results}

\textbf{Selecting appropriate model complexity}

\textbf{Posteriors}

\textbf{Realizations of the GP}

\textbf{Reduced $\chi^2$ comparison between 2-planet model and GP model}

\textbf{Periodograms of residuals}\\

Comparing the two models was done using Bayesian Information Criterion and likelihood ratios. $\chi^2$ was also computed for each model. The analysis was done by calculating the quantities for a number of the samples and creating a histogram. This allowed for the most common samples of each model to be accounted for when comparing the likelihoods of each mode. The BIC and likelihood ratios tended to favor the model that included a long term drift in the H$\alpha$ signal, this model also proved to be a lot more fragile. The model that only used a quasiperiodic model for both the RV and H$\alpha$ was consistent and gave reasonable results. Both models gave a 110-120 day period for the stellar rotation period of Kapteyn's  star.

The emcee package was used to sample the GP 8000 times, with an additional 500 steps of burn in time. 50 pairs of realizations of the RV and H$\alpha$ curves were then chosen at random from the 8000 samples, and plotted. A separate graph was also created with 3 random realization plotted in different colors to get an idea of what individual realizations look like. 

The posterior distributions of the parameters of each model were also examined to see how well the GP did at finding converged values for the parameters. For the quasiperiodic model, the distributions of the posteriors were reasonable with clear peak values and a decreasing concentration around them. \xleftarrow{}  ??Is this the best way to say this?? 

After that, a periodogram of the residuals was calculated to see if there were any significant periods present, which could indicate the presence of a planet. A period of approximately 17 days was commonly (but not exclusively) the strongest existing period in each set of residuals. However, there were no peaks with a low enough false alarm probability to indicate a planet, with no peaks reaching a false alarm probability of 10\% or lower. Additionally, the 48, 121, and 143 days periods did not show any significant peaks in the residuals of the GP.

\begin{figure} \label{fig:qsireal}
    \begin{center}
    \includegraphics[scale = 0.5]{figures/3_realization_plot.png}
    \caption{Three realizations of the Gaussian Process from the quasiperiodic model plotted with observed data. The realizations of the same color in the RV and H$\alpha$ plots correspond to the same sample.}   
    \end{center}
\end{figure}
\begin{figure} \label{fig:qsireal}
    \begin{center}
    \includegraphics[scale = 0.5]{figures/50_realization_plot.png}
    \caption{Fifty realizations of the Gaussian Process from the quasiperiodic plotted with observed data.}   
    \end{center}
\end{figure}

\begin{figure} \label{fig:qsipost}
    \begin{center}
    \includegraphics[width = 0.9\textwidth]{figures/posteriors_qsi.pdf}
    \caption{Posteriors of quasiperiodic model.}
     \end{center}
\end{figure}

\begin{figure} \label{fig:qsipdgm}
    \begin{center}
    \includegraphics[scale = 0.7]{figures/residuals.png}
    \caption{Radial velocity periodogram constructed from residuals of the quasiperiodic model. Dotted lines represent false alarm probability threshold lines of 0.10, 0.05, and 0.01. The yellow, green, and blue highlighted regions contain the 48, 121, and 143 day periods respectively. This graph contains a strongest period of 22.603 days with a false alarm probability of 0.267.}
    \end{center}
\end{figure}




\section{Discussion and conclusions}\label{sec:disc}

\textbf{What did we learn?}

\textbf{What are the implications for future exoplanet research?}\\

Acknowledgments: This research was supported by ... We thank
Dan Foreman-Mackey, Jessi Cisewski and others for their help ...

\begin{thebibliography}{99} 

% \bibitem[Aikawa et al.(1997)]{aikawa97} Aikawa, Y., Umebayashi,
% T., Nakano, T., \& Miyama, S.~M.\ 1997, \apj, 486, L51

% \bibitem[Ansdell et al.(2017)]{ansdell17} Ansdell, M., Williams,
% J.~P., Manara, C.~F., et al.\ 2017, \aj, 153, 240

\end{thebibliography}

%%%%%%%%%%%% FIGURES %%%%%%%%%%%%%%%%%

\clearpage

% Syntax for including a figure
% \begin{figure}[ht]
% \centering
% \includegraphics[width=0.90\textwidth]{fCinCO.pdf} 
% \caption{Fraction of carbon atoms incorporated in CO as a
% function of time.}
% \label{fig:fcovstime}
% \end{figure}

% Note: you can also put the figures directly in the main text

\end{document}
